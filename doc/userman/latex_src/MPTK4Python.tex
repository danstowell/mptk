\chapter{MPTK from within Python \label{GettingStartedPython}}

The functionality of MPTK can be used from within Python using the 
MPTK Python wrapper (sometimes known as ``pyMPTK''). It enables you to:
%
\begin{my_itemize}
\item decompose signals using MP or CMP, with a variety of atom types
\item reconstruct signals from books, including ones manipulated/generated from within python
\item plot convenient representations of an analysis (such as a plot of chirped atoms)
\end{my_itemize}
%
Some MPTK features are not currently present in pyMPTK:
\begin{my_itemize}
\item Creating dictionary objects/files directly
\item Some decomposition methods: GMP, LoCOMP
\end{my_itemize}

pyMPTK uses the ``NumPy'' numerical package for Python as well as Python itself.
For system requirements please see the installation sections earlier in this guide.

\section{Quick guide}
Quick summary of the API:

\begin{alltt}
	import mptk
	mptk.\textbf{loadconfig}(configpath)
	(book, residual) = mptk.\textbf{decompose}(signal, dictpath, samplerate)
	signal\_remade    = mptk.\textbf{reconstruct}(book, dictpath)

	import mptkplot
	mptkplot.\textbf{plot\_chirped}(book, samplerate)
\end{alltt}
%
Remember the ``loadconfig'' command mentioned above -- it is needed for MPTK
to know where its plugins etc are located.

For further examples, please look in the ``python'' folder in your MPTK install path, for some example scripts.
Or after running \texttt{import mptk}, use the command \texttt{help(mptk)}
(the contents of that help are included later in this manual, Section \ref{UsePythonFunc}).

