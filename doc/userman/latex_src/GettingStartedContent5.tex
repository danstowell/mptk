%--------------------------------------------------------------
%--------------------------------------------------------------
% Section  6 : MPTK command line utilities 
%--------------------------------------------------------------
%--------------------------------------------------------------
\chapter{MPTK command line utilities \label{MptkCmdLine}}

%--------------------------------------------------------------
% Section  6.1 : List of commands
%--------------------------------------------------------------
\section{List of commands}

Here is a list of the MPTK command line utilities : 
\begin{my_itemize}
	\item \textbf{mpd :} decompose a waveform signal using matching pursuit
	\item \textbf{gpd :} decompose a waveform signal using gradient pursuit
	\item \textbf{mpd\_demix :} decompose a waveform signalusing matching pursuit and a mixer matrix 
	\item \textbf{mpf :} filters the atoms contained in books
	\item \textbf{mpr :} reconstructs a signal from the atoms contained in a book
	\item \textbf{mpcat :} concatenates any number of books into one
	\item \textbf{mpview :} makes a time-frequency pixmap
\end{my_itemize}
	
%--------------------------------------------------------------
% Section  6.2 : Basic example
%--------------------------------------------------------------
\section{Basic example}

\underline{Here is a simple example on how to use MPTK with terminal commands :}

\noindent \textcolor[rgb]{0.4,0.4,0.4}{----------------------------------------------------------------------------------------------------------\newline
\textbf{mpd} command iterates Matching Pursuit on signal ``sndFileToDecomp.wav'' with dictionary    
``dictFile.xml'' and gives the resulting book ``bookFile.bin'' (and an optional residual 
signal) after N iterations or after reaching the signal-to-residual ratio SNR.\newline
----------------------------------------------------------------------------------------------------------\newline}
\$ mpd -s 10 -R 10 -d dictFile.xml sndFileToDecomp.wav bookFile.bin

\vspace{0.2 cm}

\noindent \textcolor[rgb]{0.4,0.4,0.4}{---------------------------------------------------------------------------------------------------\newline
\textbf{mpr} commands rebuilds a signal ``sndReconsFile.wav'' from the atoms contained in 
the book file ``bookFile.bin''. An optional residual ``sndResidFile.wav'' can be added.\newline                   
---------------------------------------------------------------------------------------------------\newline}
\$ mpr bookFile.bin sndReconsFile.wav sndResidFile.wav

\vspace{0.2 cm}

\noindent \emph{information : Some book examples are available under ``path\_to\_MPTK/mptk/reference/book''}

\vspace{0.1 cm}

\noindent \textcolor[rgb]{0.4,0.4,0.4}{----------------------------------------------------------------------------------------------------------------\newline
\textbf{mpf} commands filters the atoms contained in ``bookFile.bin'' (or stdin), stores those which 
satisfy the indicated properties in ``bookYes.bin'' (or stdout) and the others in ``bookNo.bin''.\newline
----------------------------------------------------------------------------------------------------------------\newline}
\$ mpf --Freq=[0:110] --len=[0:256] bookFile.bin bookYes.bin bookNo.bin

%--------------------------------------------------------------
% Section  6.3 : Finding further informations on MPTK cmd line
%--------------------------------------------------------------
\section{Finding further informations on MPTK cmd line}

\textbf{For further informations} on how to use MPTK command line utilities please refer to the 
Chapter \ref{UseCmdLine} of the User Manual.