%--------------------------------------------------------------
%--------------------------------------------------------------
% Section 3 : MPTK for Linux
%--------------------------------------------------------------
%--------------------------------------------------------------
\chapter{MPTK for Linux \label{MptkForLin}}

%--------------------------------------------------------------
% Section 3.1 : Downloading MPTK
%--------------------------------------------------------------
\section{Downloading MPTK}

The latest version of MPTK is available at (https://gforge.inria.fr/frs/?group\_id=36). Depending on the processor 
architecture of your computer, download either the 32 bits package or the 64 bits package:
\begin{my_itemize}
	\item For RedHat, Suse, Fedora, Mandriva 
	\begin{my_itemize}
		\item 32 bits : ``MPTK-binary-v.v.v-i386-Linux.rpm''
		\item 64 bits : ``MPTK-binary-v.v.v-x86\_64-Linux.rpm''
	\end{my_itemize}
	\item For Debian, Knoppix, Ubuntu 
	\begin{my_itemize}
		\item 32 bits : ``MPTK-binary-v.v.v-i386-Linux.deb''
		\item 64 bits : ``MPTK-binary-v.v.v-x86\_64-Linux.deb"
	\end{my_itemize}
\end{my_itemize}

\noindent \emph{\underline{Hint :} To find the processor architecture of your computer :}
\begin{my_itemize}
	\item \emph{Open a terminal command and use the following command : \textbf{uname -m}}
	\begin{my_itemize}
		\item \emph{If the answer is ``i386'' then your OS is 32 bits}
		\item \emph{If the answer is ``x86\_64'' then your OS is 64 bits}
	\end{my_itemize}
\end{my_itemize}

%--------------------------------------------------------------
% Section 3.1 : Downloading MPTK
%--------------------------------------------------------------
\section{Obtaining additional required packages}

Two additional packages are needed. Their installation require administrator privileges on the machine. 
The ``sudo'' command may ask you to input administrator password :
\begin{my_itemize}
	\item Libsndfile (tested with version 1.0.23) pre compiled library
	\item FFTW (tested with version 3.2.2) pre compiled library
	\item Optional, but required to enable the Python wrapper pyMPTK:
	\begin{my_itemize}
		\item Python (tested with version 2.7)
		\item NumPy (tested with version 1.5)
		\item Matplotlib (tested with version 1.0)
	\end{my_itemize}
\end{my_itemize}

\noindent \emph{You can see below some examples about how to download those libraries using terminal :}

\vspace{0.3 cm}

\begin{tabular}{|l|}
	\hline
	\rowcolor[gray]{0.35} \textcolor{white}{\emph{Ubuntu}} \\
		\texttt{
			\emph{sudo apt-get install libsndfile1-dev libfftw3-dev}
		}
	\\
		\texttt{
			\emph{sudo apt-get install python-dev python-numpy python-matplotlib}
		}
	\\
	\hline
	\rowcolor[gray]{0.35} \textcolor{white}{\emph{Fedora}} \\
		 \texttt{
		 	\emph{sudo yum install libsndfile-devel fftw-devel fftw-static}
		}
	\\
		 \texttt{
		 	\emph{sudo yum install python numpy python-matplotlib}
		}
	\\
	\hline
	\rowcolor[gray]{0.35} \textcolor{white}{\emph{Mandriva}} \\
		\texttt{
			\emph{sudo smart install libsndfile1-dev libfftw3-dev}
		}
	\\
		\texttt{
			\emph{sudo smart install python-dev python-numpy python-matplotlib}
		}
	\\
	\hline
\end{tabular}

%--------------------------------------------------------------
% Section 3.2 : Installing MPTK
%--------------------------------------------------------------
\section{Installing MPTK}

Depending on the type of Linux you have there ares two ways to install the packages : 
\begin{my_itemize}
	\item for ``rpm'' package : \textcolor[rgb]{0.4,0.4,0.4}{\emph{rpm -ivh MPTK-binary-v.v.v-(i386/x86\_64)-Linux.rpm}}
	\item for ``deb'' package : \textcolor[rgb]{0.4,0.4,0.4}{\emph{dpkg -i MPTK-binary-v.v.v-(i386/x86\_64)-Linux.deb}}
\end{my_itemize}
%--------------------------------------------------------------
% Section 3.3 : Configuring the path
%--------------------------------------------------------------
\section{Configuring the path}

An environment variable called \begin{footnotesize}MPTK\_CONFIG\_FILENAME\end{footnotesize} needs to be set, either 
temporarily, either permanently, with the path of the ``path.xml'' file located in the \emph{``path\_to\_MPTK/mptk''} directory. This 
file defines the environment paths that MPTK needs to work properly.

% Section 3.3.1 : Temporary path configuration

\subsection{Temporary path configuration}
	
Here is the way to temporarily configure the \begin{footnotesize}MPTK\_CONFIG\_FILENAME\end{footnotesize} environment variable. 
\textbf{Warning :} This is a temporary setting and it needs to be done at each reset of the computer.
\begin{my_itemize}
	\item \emph{With Bash shell :}
	\begin{my_itemize}
		\item \textcolor[rgb]{0.4,0.4,0.4}{\emph{export \begin{footnotesize}MPTK\_CONFIG\_FILENAME\end{footnotesize} =``path\_to\_MPTK/mptk/path.xml''}}
	\end{my_itemize}
	\item \emph{With C-shell :}
	\begin{my_itemize}
		\item \textcolor[rgb]{0.4,0.4,0.4}{\emph{setenv \begin{footnotesize}MPTK\_CONFIG\_FILENAME\end{footnotesize} ``path\_to\_MPTK/mptk/path.xml''}}
	\end{my_itemize}
	\item \emph{You can check if the environment variable is correctly set with :}
	\begin{my_itemize}
		\item \textcolor[rgb]{0.4,0.4,0.4}{\emph{echo \$\begin{footnotesize}MPTK\_CONFIG\_FILENAME\end{footnotesize}}}
	\end{my_itemize}
\end{my_itemize}

% Section 3.3.1 : Temporary path configuration

\subsection{Permanent path configuration}

In order to permanently configure the \begin{footnotesize}MPTK\_CONFIG\_FILENAME\end{footnotesize} environment variable, 
add  the bash shell (or the C-shell) configuration line to the ``.bashrc'' (or the ``.cshrc'') file situated under your home directory.

% Section 3.3.1 : Temporary path configuration

\subsection{Matlab path configuration}

When launching Matlab, the user needs to configure Matlab to work with MPTK:
\begin{my_itemize}
	\item Configure the working path either by: 
	\begin{my_itemize}
		\item Selecting the current folder as \textcolor[rgb]{0.4,0.4,0.4}{\emph{``path\_to\_MPTK/mptk/matlab''}}
		\item Adding the working path using \textcolor[rgb]{0.4,0.4,0.4}{\emph{addpath(``path\_to\_MPTK/mptk/matlab'')}}
	\end{my_itemize}
\end{my_itemize}
