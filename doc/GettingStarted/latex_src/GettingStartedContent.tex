%--------------------------------------------------------------
%--------------------------------------------------------------
% Section 1 : Introduction
%--------------------------------------------------------------
%--------------------------------------------------------------
\chapter{Introduction}

%--------------------------------------------------------------
% Section 1.1 : Learning about MPTK
%--------------------------------------------------------------
\section{Learning about MPTK}
	
The Matching Pursuit Tool Kit (MPTK) provides a fast implementation of the Matching Pursuit algorithm for 
the sparse decomposition of multichannel signals such as audio signals. It comprises a library, standalone 
utilities and Matlab scripts.

MPTK provides an implementation of Matching Pursuit which is: 

\begin{my_itemize}
	\item \textbf{FAST:} e.g., extract 1.5 million atoms from a 1 hour long, 16kHz audio signal (15dB extracted) in 0.25x 
	real time on a Pentium IV@2.4GHz, out of a dictionary of 178M Gabor atoms. Such incredible speed makes it 
	possible to process ``real world'' signals.
	\item \textbf{FLEXIBLE:} multi-channel, various signal input formats, flexible syntax to describe the dictionaries 
	$\mapsto$ reproducible results, cleaner experiments. 
	\item OPEN: modular architecture $\mapsto$ add your own atoms ! Free distribution under the GPL. 
\end{my_itemize}

MPTK is mainly developed and maintained within the METISS Research Group \linebreak
(http://www.irisa.fr/metiss/) on audio signal processing, at the INRIA Research 
Institute (http://www.irisa.fr or http://www.inria.fr/rennes) in Rennes, France.

%--------------------------------------------------------------
% Section 1.2 : Reading this document
%--------------------------------------------------------------
\section{Reading this document}

This document describes the basic principles about how to download, install and use the Matching Pursuit Tool Kit. 
It is divided into two major sections : 

\begin{my_itemize}
	\item Downloading and installing MPTK depending on the system OS 
	\item Understanding the basic usage of MPTK with command line tools
	\item Understanding the basic usage of MPTK within Matlab
\end{my_itemize}

If you need more specific details about any part of this Getting Started document, please refer to the following chapters 
of the \textbf{User manual} document  :

\begin{my_itemize}
	\item \textbf{Chapter 10} for how to pre-build, build and install MPTK from the source files
	\item \textbf{Chapter 11} for how to use MPTK with command line tools
	\item \textbf{Chapter 13} for how to use MPTK within Matlab
\end{my_itemize}

\noindent The User manual document can be downloaded here : 
\begin{my_itemize}
	\item https://gforge.inria.fr/docman/?group\_id=36
\end{my_itemize}

\clearpage

%--------------------------------------------------------------
%--------------------------------------------------------------
%--------------------------------------------------------------
% Part 1 : Installing  MPTK
%--------------------------------------------------------------
%--------------------------------------------------------------
%--------------------------------------------------------------
\mypart{Installing MPTK}

%--------------------------------------------------------------
%--------------------------------------------------------------
% Section 2 : MPTK for Windows
%--------------------------------------------------------------
%--------------------------------------------------------------
\chapter{MPTK for Windows \label{MptkForWin}}

%--------------------------------------------------------------
% Section 2.1 : Downloading MPTK
%--------------------------------------------------------------
\section{Downloading MPTK}

The latest version of MPTK is available at (https://gforge.inria.fr/frs/?group\_id=36). Depending on the processor architecture of 
your computer, you will have to download either the 32 bits package or the 64 bits package:
\begin{my_itemize}
	\item For Windows 32 bits : ``MPTK-binary-v.v.v-i386-Windows.exe''
	\item For Windows 64 bits : ``MPTK-binary-v.v.v-x86\_64-Windows.exe''
\end{my_itemize}

\noindent \emph{\underline{Hint :} To find the processor architecture of your computer:}
\begin{my_itemize}
	\item \emph{Open a terminal command using :}
	\begin{my_itemize}
		\item \emph{Start $\mapsto$ All Programs $\mapsto$ Accessories $\mapsto$ Command Prompt}
	\end{my_itemize}
	\item \emph{Use the following command : \textcolor[rgb]{0.4,0.4,0.4}{echo \%PROCESSOR\_ARCHITECTURE\%}}
	\begin{my_itemize}
		\item \emph{If the answer is ``x86'' then your OS is 32 bits}
		\item \emph{If the answer is ``AMD64'' then your OS is 64 bits}
	\end{my_itemize}
\end{my_itemize}

%--------------------------------------------------------------
% Section 2.2 : Installing MPTK
%--------------------------------------------------------------
\section{Installing MPTK}

When double clicking the executable ``MPTK-binary-v.v.v.-(i386/x86\_64)-Windows.exe'':
\begin{enumerate}
	\item Accept the terms of the licence agreement
	\item Select the path folder where to install MPTK (we'll call it  \emph{``path\_to\_MPTK''})
	\begin{my_itemize}
		\item We suggest to use the default folder : ``C:\textbackslash Program Files\textbackslash MPTK''
	\end{my_itemize}
	\item Finish the installation
\end{enumerate}	

\begin{figure}[H]
	\begin{minipage}[t]{.4\linewidth}
   	 	\begin{center}
			\includegraphics[width = 7cm, height=5cm]{\relativePath WindowsLicence.eps}
			\textit{\caption{\label{WindowsLicence} Licence agreement}}
		\end{center}
	\end{minipage}
	\hfill
	\begin{minipage}[t]{.4\linewidth}
		\begin{center}
			\includegraphics[width = 7cm, height=5cm]{\relativePath PathSelection.eps}
			\textit{\caption{\label{PathSelection}Path folder selection}}
		\end{center}
	\end{minipage}
\end{figure}

%--------------------------------------------------------------
% Section 2.3 : Configuring the path
%--------------------------------------------------------------
\section{Configuring the path}

An environment variable called \begin{footnotesize}MPTK\_CONFIG\_FILENAME\end{footnotesize} needs to be set, either temporarily, or permanently, 
with the path of the ``path.xml'' file located in the \emph{``path\_to\_MPTK/mptk''} directory. This file defines the environment 
paths that MPTK needs to work correctly.


% Section 2.3.1 : Temporary path configuration

\subsection{Temporary path configuration}

Here is the way to temporarily configure the \begin{footnotesize}MPTK\_CONFIG\_FILENAME\end{footnotesize} environment variable. 
\textbf{Warning} : This is a temporary setting and it needs to be done at each reset of the computer.

\begin{my_itemize}
	\item Open a terminal command using :
	\begin{my_itemize}
		\item Start $\mapsto$ All Programs $\mapsto$ Accessories $\mapsto$ Command Prompt
	\end{my_itemize}
	\item Use the command : \textcolor[rgb]{0.4,0.4,0.4}{\emph{set \begin{footnotesize}MPTK\_CONFIG\_FILENAME\end{footnotesize} = path\_to\_MPTK/mptk/path.xml}}
\end{my_itemize}

\begin{figure}[H]
   	 \begin{center}
		\includegraphics[width = 7cm, height=3cm]{\relativePath CommandPrompt.eps}
		\textit{\caption{\label{CommandPrompt} Filled command prompt}}
	\end{center}
\end{figure}

% Section 2.3.2 : Permanent path configuration

\subsection{Permanent path configuration}
 
\noindent Here is the way to permanently configure the \begin{footnotesize}MPTK\_CONFIG\_FILENAME\end{footnotesize} environment variable
\begin{my_itemize}
	\item Check if the environment variable is correctly set with : echo \begin{footnotesize}\%MPTK\_CONFIG\_FILENAME\%\end{footnotesize}
	\item Open the environment variable configuration panel situated under  :
	\begin{my_itemize}
		\item Start $\mapsto$ Config panel $\mapsto$ System $\mapsto$ Advanced $\mapsto$ Environment variables
	\end{my_itemize}
	\item Add a new user variable with :
	\begin{my_itemize}
		\item Name : \begin{footnotesize}MPTK\_CONFIG\_FILENAME\end{footnotesize}
		\item Value : \emph{path\_to\_MPTK/mptk/path.xml}
	\end{my_itemize}
\end{my_itemize}

\begin{figure}[H]
   	 \begin{center}
		\includegraphics[width = 5cm, height=4cm]{\relativePath EnvironmentConfiguration.eps}
		\textit{\caption{\label{EnvironmentConfiguration} Environment variable configuration}}
	\end{center}
\end{figure}

% Section 2.3.2 : Matlab path configuration

\subsection{Matlab path configuration}

When launching Matlab, the user needs to configure Matlab to work with MPTK:
\begin{my_itemize}
	\item Configure the working path either by: 
	\begin{my_itemize}
		\item Selecting the current folder as \textcolor[rgb]{0.4,0.4,0.4}{``path\_to\_MPTK/mptk/matlab''}
		\item Adding the working path using \textcolor[rgb]{0.4,0.4,0.4}{addpath(``path\_to\_MPTK/mptk/matlab'')}
	\end{my_itemize}
\end{my_itemize}


%--------------------------------------------------------------
%--------------------------------------------------------------
% Section 3 : MPTK for Linux
%--------------------------------------------------------------
%--------------------------------------------------------------
\chapter{MPTK for Linux \label{MptkForLin}}

%--------------------------------------------------------------
% Section 3.1 : Downloading MPTK
%--------------------------------------------------------------
\section{Downloading MPTK}

The latest version of MPTK is available at (https://gforge.inria.fr/frs/?group\_id=36). Depending on the processor 
architecture of your computer, download either the 32 bits package or the 64 bits package:
\begin{my_itemize}
	\item For RedHat, Suse, Fedora, Mandriva 
	\begin{my_itemize}
		\item 32 bits : ``MPTK-binary-v.v.v-i386-Linux.rpm''
		\item 64 bits : ``MPTK-binary-v.v.v-x86\_64-Linux.rpm''
	\end{my_itemize}
	\item For Debian, Knoppix, Ubuntu 
	\begin{my_itemize}
		\item 32 bits : ``MPTK-binary-v.v.v-i386-Linux.deb''
		\item 64 bits : ``MPTK-binary-v.v.v-x86\_64-Linux.deb"
	\end{my_itemize}
\end{my_itemize}

\noindent \emph{\underline{Hint :} To find the processor architecture of your computer :}
\begin{my_itemize}
	\item \emph{Open a terminal command and use the following command : \textbf{uname -m}}
	\begin{my_itemize}
		\item \emph{If the answer is ``i386'' then your OS is 32 bits}
		\item \emph{If the answer is ``x86\_64'' then your OS is 64 bits}
	\end{my_itemize}
\end{my_itemize}

%--------------------------------------------------------------
% Section 3.1 : Downloading MPTK
%--------------------------------------------------------------
\section{Obtaining additional required packages}

Two additional packages are needed. Their installation require administrator privileges on the machine. 
The ``sudo'' command may ask you to input administrator password :
\begin{my_itemize}
	\item Libsndfile (tested with version 1.0.23) pre compiled library
	\item FFTW (tested with version 3.2.2) pre compiled library
\end{my_itemize}

\noindent \emph{You can see below some examples about how to download those libraries using terminal :}

\vspace{0.3 cm}

\begin{tabular}{|c|c|c|}
	\hline
	\rowcolor[gray]{0.35} \textcolor{white}{\emph{Ubuntu}} &  \textcolor{white}{\emph{Fedora}} &  \textcolor{white}{\emph{Mandriva}} \\
	\hline \hline
		\begin{tiny}
			\emph{sudo apt-get install -y -qq libsndfile1-dev}
		\end{tiny}
		 & 
		 \begin{tiny}
		 	\emph{sudo yum -y -qq install libsndfile-devel}
		\end{tiny}
		& 
		\begin{tiny}
			\emph{sudo smart install -y -qq libsndfile1-dev}
		\end{tiny} \\
		\begin{tiny}
			\emph{sudo apt-get install -y -qq libfftw3-dev}
		\end{tiny}
		 & 
		 \begin{tiny}
		 	\emph{sudo yum -y -qq install fftw-devel}
		\end{tiny}
		& 
		\begin{tiny}
			\emph{sudo smart install -y -qq libfftw3-dev}
		\end{tiny} \\
		& 
		\begin{tiny}
			\emph{sudo yum -y -qq install fftw-static}
		\end{tiny}
		& \\
	\hline
\end{tabular}

%--------------------------------------------------------------
% Section 3.2 : Installing MPTK
%--------------------------------------------------------------
\section{Installing MPTK}

Depending on the type of Linux you have there ares two ways to install the packages : 
\begin{my_itemize}
	\item for ``rpm'' package : \textcolor[rgb]{0.4,0.4,0.4}{\emph{rpm -ivh MPTK-binary-v.v.v-(i386/x86\_64)-Linux.rpm}}
	\item for ``deb'' package : \textcolor[rgb]{0.4,0.4,0.4}{\emph{dpkg -i MPTK-binary-v.v.v-(i386/x86\_64)-Linux.deb}}
\end{my_itemize}
%--------------------------------------------------------------
% Section 3.3 : Configuring the path
%--------------------------------------------------------------
\section{Configuring the path}

An environment variable called \begin{footnotesize}MPTK\_CONFIG\_FILENAME\end{footnotesize} needs to be set, either 
temporarily, either permanently, with the path of the ``path.xml'' file located in the \emph{``path\_to\_MPTK/mptk''} directory. This 
file defines the environment paths that MPTK needs to work properly.

% Section 3.3.1 : Temporary path configuration

\subsection{Temporary path configuration}
	
Here is the way to temporarily configure the \begin{footnotesize}MPTK\_CONFIG\_FILENAME\end{footnotesize} environment variable. 
\textbf{Warning :} This is a temporary setting and it needs to be done at each reset of the computer.
\begin{my_itemize}
	\item \emph{With Bash shell :}
	\begin{my_itemize}
		\item \textcolor[rgb]{0.4,0.4,0.4}{\emph{export \begin{footnotesize}MPTK\_CONFIG\_FILENAME\end{footnotesize} =``path\_to\_MPTK/mptk/path.xml''}}
	\end{my_itemize}
	\item \emph{With C-shell :}
	\begin{my_itemize}
		\item \textcolor[rgb]{0.4,0.4,0.4}{\emph{setenv \begin{footnotesize}MPTK\_CONFIG\_FILENAME\end{footnotesize} ``path\_to\_MPTK/mptk/path.xml''}}
	\end{my_itemize}
	\item \emph{You can check if the environment variable is correctly set with :}
	\begin{my_itemize}
		\item \textcolor[rgb]{0.4,0.4,0.4}{\emph{echo \$\begin{footnotesize}MPTK\_CONFIG\_FILENAME\end{footnotesize}}}
	\end{my_itemize}
\end{my_itemize}

% Section 3.3.1 : Temporary path configuration

\subsection{Permanent path configuration}

In order to permanently configure the \begin{footnotesize}MPTK\_CONFIG\_FILENAME\end{footnotesize} environment variable, 
add  the bash shell (or the C-shell) configuration line to the ``.bashrc'' (or the ``.cshrc'') file situated under your home directory.

% Section 3.3.1 : Temporary path configuration

\subsection{Matlab path configuration}

When launching Matlab, the user needs to configure Matlab to work with MPTK:
\begin{my_itemize}
	\item Configure the working path either by: 
	\begin{my_itemize}
		\item Selecting the current folder as \textcolor[rgb]{0.4,0.4,0.4}{\emph{``path\_to\_MPTK/mptk/matlab''}}
		\item Adding the working path using \textcolor[rgb]{0.4,0.4,0.4}{\emph{addpath(``path\_to\_MPTK/mptk/matlab'')}}
	\end{my_itemize}
\end{my_itemize}

%--------------------------------------------------------------
%--------------------------------------------------------------
% Section 4 :  MPTK for Mac OS
%--------------------------------------------------------------
%--------------------------------------------------------------
\chapter{MPTK for Mac OS \label{MptkForMac}}

%--------------------------------------------------------------
% Section 4.1 : Downloading MPTK
%--------------------------------------------------------------
\section{Downloading MPTK}

The latest version of MPTK is available at (https://gforge.inria.fr/frs/?group\_id=36). Depending on the processor architecture 
of your computer, you will have to download either the 32 bits package or the 64 bits package:
\begin{my_itemize}
	\item For Mac 32 bits : ``MPTK-binary-v.v.v-i386-Mac.exe''
	\item For Mac 64 bits : `MPTK-binary-v.v.v-x86\_64-Mac.exe''
\end{my_itemize}

\noindent \emph{\underline{Hint :} To find the processor architecture of your computer :}
\begin{my_itemize}
	\item \emph{Open a terminal command and use the following command : \textbf{uname -m}}
	\begin{my_itemize}
		\item \emph{If the answer is ``i386'' then your OS is 32 bits}
		\item \emph{If the answer is ``x86\_64'' then your OS is 64 bits}
	\end{my_itemize}
\end{my_itemize}

%--------------------------------------------------------------
% Section 4.2 : Obtaining additional required packages
%--------------------------------------------------------------
\section{Obtaining additional required packages}

Two additional packages are needed. Their installation require administrator privileges on the machine. 
The ``sudo''  command may ask you to input administrator password :
\begin{my_itemize}
	\item Libsndfile (tested with version 1.0.23) pre compiled library
	\item FFTW (tested with version 3.2.2) pre compiled library
\end{my_itemize}

\noindent You can see below some examples about how to download those libraries using terminal: 

\vspace{0.3 cm}

\begin{center}
	\begin{tabular}{|c|}
		\hline
		\rowcolor[gray]{0.35} \textcolor{white}{\emph{Mac}} \\
		\hline \hline
			\begin{tiny}
				\emph{sudo /opt/local/bin/port install libsndfile +universal}
			\end{tiny} \\
			\begin{tiny}
				\emph{sudo /opt/local/bin/port install fftw-3 +universal}
			\end{tiny} \\
		\hline
	\end{tabular}
\end{center}

\noindent \emph{\underline{Hint :} We suggest to use the ``port'' command from MacPorts because the command 
``+universal'' allows to retrieve libraries which are compatible with both system architectures (32 bits and 64 bits). 
The package is available at http://www.macports.org/install.php}

%--------------------------------------------------------------
% Section 4.3 : Installing MPTK
%--------------------------------------------------------------
\section{Installing MPTK}

When double clicking the executable \textbf{``MPTK-binary-v.v.v.-(i386/x86\_64)-Mac.dmg''}:
\begin{enumerate}
	\item Accept the terms of the licence agreements
	\item Accept the path folder where to install MPTK
	\begin{my_itemize}
		\item The default and unique folder is : ``/usr/local/''
	\end{my_itemize}
	\item Finish the installation
\end{enumerate}

\begin{figure}[H]
	\begin{minipage}[t]{.4\linewidth}
   	 	\begin{center}
			\includegraphics[width = 7cm, height=5cm]{\relativePath MacLicence.eps}
			\textit{\caption{\label{MacLicence} Licence agreement}}
		\end{center}
	\end{minipage}
	\hfill
	\begin{minipage}[t]{.4\linewidth}
		\begin{center}
			\includegraphics[width = 7cm, height=5cm]{\relativePath MacPathSelection.eps}
			\textit{\caption{\label{MacPathSelection}Path folder selection}}
		\end{center}
	\end{minipage}
\end{figure}


 %--------------------------------------------------------------
% Section 4.4 : Configuring the path
%--------------------------------------------------------------
\section{Configuring the path}

An environment variable called \begin{footnotesize}MPTK\_CONFIG\_FILENAME\end{footnotesize} needs to be set, 
either temporary, either permanently, with the path of the �path.xml� file located in the \emph{``path\_to\_MPTK/mptk''} directory. 
This file defines the environment paths that MPTK needs to work properly.

% Section 4.4.1 : Configuring the path

\subsection{Temporary path configuration}

Here is the way to temporary configure the \begin{footnotesize}MPTK\_CONFIG\_FILENAME\end{footnotesize} environment variable. 
\textbf{Warning :} This is a temporary setting and it needs to be done at each reset of the computer.

\begin{my_itemize}
	\item \emph{With Bash shell :}
	\begin{my_itemize}
		\item \textcolor[rgb]{0.4,0.4,0.4}{export \begin{footnotesize}MPTK\_CONFIG\_FILENAME\end{footnotesize} =``/usr/local/mptk/path.xml''}
	\end{my_itemize}
	\item \emph{With C-shell :}
	\begin{my_itemize}
		\item \textcolor[rgb]{0.4,0.4,0.4}{setenv \begin{footnotesize}MPTK\_CONFIG\_FILENAME\end{footnotesize} ``/usr/local/mptk/path.xml''}
	\end{my_itemize}
	\item \emph{You can check if the environment variable is correctly set with :}
	\begin{my_itemize}
		\item \textcolor[rgb]{0.4,0.4,0.4}{echo \$\begin{footnotesize}MPTK\_CONFIG\_FILENAME\end{footnotesize}}
	\end{my_itemize}
\end{my_itemize}

% Section 4.4.2 : Permanent path configuration

\subsection{Permanent path configuration}

In order to permanently configure the \begin{footnotesize}MPTK\_CONFIG\_FILENAME\end{footnotesize} environment variable, 
add  the bash shell (or the C-shell) configuration sentence to the ``.bashrc'' (or the ``.cshrc'') file situated under your home directory.

% Section 4.4.2 : Matlab path configuration

\subsection{Matlab path configuration}

When launching Matlab, the user needs to configure Matlab to work with MPTK:
\begin{my_itemize}
	\item Configure the working path either by : 
	\begin{my_itemize}
		\item Selecting the current folder as \textcolor[rgb]{0.4,0.4,0.4}{``/usr/local/mptk/matlab''}
		\item Adding the working path using \textcolor[rgb]{0.4,0.4,0.4}{addpath(``/usr/local/mptk/matlab'')}
	\end{my_itemize}
\end{my_itemize}

%--------------------------------------------------------------
%--------------------------------------------------------------
%--------------------------------------------------------------
% Part 2 : Using  MPTK
%--------------------------------------------------------------
%--------------------------------------------------------------
%--------------------------------------------------------------
\mypart{Using MPTK}


%--------------------------------------------------------------
%--------------------------------------------------------------
% Section  5 : MPTK from within Matlab
%--------------------------------------------------------------
%--------------------------------------------------------------
\chapter{MPTK from within Matlab \label{MptkWithMatlab}}

In the following, we assume that MPTK has been installed correctly, that the system path configuration has been set and that Matlab is installed.

%--------------------------------------------------------------
% Section  5.1 : Getting Started
%--------------------------------------------------------------
\section{Getting Started}

\textcolor[rgb]{0.4,0.4,0.4}{\emph{\textbf{GettingStarted} command is used for a better understanding of MPTK functionalities. This script is divided in two parts. 
The first part consist on retrieving the environment informations and the available plugins. The second part consist on describing 
several tutorials about MPTK utilities.}}

\vspace{0.3 cm}

\noindent \underline{\emph{1st part description : Example of available plugins and dictionaries}}

\noindent Here is the list of types of atoms available in MPTK plugins:
\begin{center}
	\begin{tabular}{ll}
			- anywave	 & - anywavehilbert \\
			- constant	 & - dirac \\
			- gabor	 & - harmonic \\
			- mclt	 & - mdct \\
			- mdst	 & - nyquist \\
	\end{tabular}
\end{center}

\noindent As well as information on the path where reference files can be found:
\begin{center}
	- /usr/local/mptk/reference
\end{center}

which can be used to find examples of dictionaries:
\begin{center}
	\begin{tabular}{lll}
			dic\_anywave.xml & dic\_constant.xml & dic\_harmonic.xml \\
			dic\_mdst.xml & dic\_anywave\_modifie.xml & dic\_dirac.xml \\
			dic\_mclt.xml & dic\_nyquist.xml & dic\_chirp.xml \\
			dic\_gabor\_two\_scales.xml & dic\_mdct\_two\_scales.xml & dico\_test.xml \\
	\end{tabular}
\end{center}

\vspace{0.3 cm}

\noindent \underline{\emph{2nd part description : Description of the tutorials}}

\noindent There are several tutorials on using MPTK4Matlab:
\begin{my_enumerate}
	\item Dictionaries
	\item Books
	\item Running Matching Pursuit Toolkit
	\item Multichannel decompositions (in preparation) 
	\item Anywave atoms (in preparation)
	\item Demixing pursuit (in preparation)
\end{my_enumerate}

\noindent \underline{\emph{Dictionaries :}} How to read (\emph{dictread}), create (\emph{dictwrite}) a dictionary description \newline
\noindent \underline{\emph{Books :}} What is a book (storage format for sparse signal representations) and how to read 
(\emph{bookread}), save (\emph{bookwrite}) or plot any book (\emph{bookplot}, \emph{bookover}) \newline
\noindent \underline{\emph{Running MPTK :}} Procedure to follow if you want to decompose a signal :
\begin{my_itemize}
	\item Read a signal (\emph{sigread})
	\item Read a dictionary (\emph{dictread}) 
	\item Decompose the signal (\emph{mpdecomp})
\end{my_itemize}

%--------------------------------------------------------------
% Section  5.2 : Getting the environment information
%--------------------------------------------------------------
\section{Getting the environment information}

\textcolor[rgb]{0.4,0.4,0.4}{\emph{\textbf{getmptkinfo} command is launched under the GettingStarted script, and is used to retrieve 
the environment datas, such as : }}

\begin{my_itemize}
	\item \textcolor[rgb]{0.4,0.4,0.4}{The plugins atoms names available}
	\item \textcolor[rgb]{0.4,0.4,0.4}{The block parameters needed to correctly define each atom}
	\item \textcolor[rgb]{0.4,0.4,0.4}{The windows names that can be used to through the signal waveform}
	\item \textcolor[rgb]{0.4,0.4,0.4}{The example or default environment paths}
\end{my_itemize}

\noindent $\mapsto$ info = getmptkinfo();

%--------------------------------------------------------------
% Section  5.3 : Reading a signal
%--------------------------------------------------------------
\section{Reading a signal}

\textcolor[rgb]{0.4,0.4,0.4}{\emph{\textbf{sigread} command reads an imports signal ``exampleSignal'' of any format supported by  libsndfile library 
to Matlab and gives a matrix ``signal'' (numSamples x numChans) and the sampling frequency of the read signal ``sampleRate''.}}

\vspace{0.2 cm}

\noindent $\mapsto$ [signal sampleRate] = sigread(mptkInfo.path.exampleSignal);

%--------------------------------------------------------------
% Section  5.4 : Reading a dictionary
%--------------------------------------------------------------
\section{Reading a dictionary}

\textcolor[rgb]{0.4,0.4,0.4}{\emph{\textbf{dictread} command imports a dictionary description ``defaultDict'' to Matlab and gives a dictionary 
description with the following structure :
		dict.block\{i\} = block       where, for example
			block.type = `dirac'}}

\vspace{0.2 cm}

\noindent $\mapsto$ dict = dictread(mptkInfo.path.defaultDict);

%--------------------------------------------------------------
% Section  5.5 : Decomposing a signal
%--------------------------------------------------------------
\section{Decomposing a signal}

\textcolor[rgb]{0.4,0.4,0.4}{\emph{\textbf{mpdecomp} command decompose a signal ``Signal'' using its sampling frequency ``sampleRate'', 
a dictionary structure ``dict'' performing ``numIter'' iterations and  gives the resulting decomposition ``book'', the ``residual'' obtained after 
the iterations and ``decay'', a vector with the energy of the residual after each iteration.}}

\vspace{0.2 cm}

\noindent $\mapsto$ [book residual decay] = mpdecomp(signal,sampleRate,dict,numIter);

%--------------------------------------------------------------
% Section  5.6 : Plotting a book
%--------------------------------------------------------------
\section{Plotting a book}

\textcolor[rgb]{0.4,0.4,0.4}{\emph{\textbf{bookover} plots the given ``book'' over a STFT spectrogram of the given ``Signal'' for  channel ``numChan'' 
(or 1 for default). The book and/or the signal can be given as  filenames (WAV format for the signal).}}

\vspace{0.2 cm}

\noindent $\mapsto$ figure(5);bookover(book,signal);

%--------------------------------------------------------------
% Section  5.7 : Reconstructing a signal
%--------------------------------------------------------------
\section{Reconstructing a signal}

\textcolor[rgb]{0.4,0.4,0.4}{\emph{\textbf{mprecons} reconstructs the signal from the given ``book''.}}

\vspace{0.2 cm}

\noindent $\mapsto$ sigrec = mprecons(book);

\vspace{0.3 cm}

\textbf{For further information} on how to use MPTK Matlab functionalities please refer to the Chapter ``How to use MPTK with Matlab'' 
of the Usermanual documentation situated under the gforge website (https://gforge.inria.fr/docman/?group\_id=36) or 
under the documentation directory (MPTK UserManual).


%--------------------------------------------------------------
%--------------------------------------------------------------
% Section  6 : MPTK command line utilities 
%--------------------------------------------------------------
%--------------------------------------------------------------
\chapter{MPTK command line utilities \label{MptkCmdLine}}

\underline{Here is a simple example on how to use MPTK with terminal commands :}

\noindent \textcolor[rgb]{0.4,0.4,0.4}{----------------------------------------------------------------------------------------------------------\newline
\textbf{mpd} command iterates Matching Pursuit on signal ``sndFileToDecomp.wav'' with dictionary    
``dictFile.xml'' and gives the resulting book ``bookFile.bin'' (and an optional residual 
signal) after N iterations or after reaching the signal-to-residual ratio SNR.\newline
----------------------------------------------------------------------------------------------------------\newline}
\$ mpd -s 10 -R 10 -d /dictionary/dictFile.xml sndFileToDecomp.wav bookFile.bin

\vspace{0.2 cm}

\noindent \textcolor[rgb]{0.4,0.4,0.4}{---------------------------------------------------------------------------------------------------\newline
\textbf{mpr} commands rebuilds a signal ``sndReconsFile.wav'' from the atoms contained in 
the book file ``bookFile.bin''. An optional residual ``sndResidFile.wav'' can be added.\newline                   
---------------------------------------------------------------------------------------------------\newline}
\$ mpr bookFile.bin sndReconsFile.wav sndResidFile.wav

\vspace{0.2 cm}

\noindent \emph{information : Some book examples are available under ``path\_to\_MPTK/mptk/reference/book''}

\vspace{0.1 cm}

\noindent \textcolor[rgb]{0.4,0.4,0.4}{----------------------------------------------------------------------------------------------------------------\newline
\textbf{mpf} commands filters the atoms contained in ``bookFile.bin'' (or stdin), stores those which 
satisfy the indicated properties in ``bookYes.bin'' (or stdout) and the others in ``bookNo.bin''.\newline
----------------------------------------------------------------------------------------------------------------\newline}
\$ mpf --Freq=[0:110] --len=[0:256] bookFile.bin bookYes.bin bookNo.bin

\vspace{0.3 cm}

\textbf{For further information} on how to use MPTK Matlab functionalities please refer to the Chapter ``How to use MPTK command lines'' 
of the Usermanual documentation situated under the gforge website (https://gforge.inria.fr/docman/?group\_id=36) or under the 
documentation directory (MPTK UserManual).

%--------------------------------------------------------------
%--------------------------------------------------------------
% Section  7 : Source installation
%--------------------------------------------------------------
%--------------------------------------------------------------
\chapter{Source installation}

The source installation archive file (https://gforge.inria.fr/frs/?group\_id=36) is available under gforge website . 
This kind of installation is recommended for :
\begin{my_itemize}
	\item Whoever have problems installing the binary files
	\item Whoever wants to help developing MPTK and/or adding new functionalities to MPTK
\end{my_itemize}

For further information on how to pre-build with cmake, build and install MPTK from the source files, please refer to the chapter 
``Download \& install from source'' of the Usermanual documentation situated under gforge website (https://gforge.inria.fr/docman/?group\_id=36) 
or under the documentation directory (MPTK UserManual).

%--------------------------------------------------------------
%--------------------------------------------------------------
% Section  8 : Help, contact and forums
%--------------------------------------------------------------
%--------------------------------------------------------------
\chapter{Help, contact and forums}

\noindent \underline{\emph{If you need help with the software:}}
\begin{my_enumerate}
	\item Check if a more recent release fixes your problem (https://gforge.inria.fr/frs/?group\_id=36)
	\item Check if somebody else has had a samilar problem and if a fix exist on the help forum (https://gforge.inria.fr/forum/forum.php?forum\_id=109)
	\item If not, post a message on the help forum
\end{my_enumerate}

\noindent \underline{\emph{If you need documentation about the software:}}

\vspace{0.2 cm}

Some articles exposing scientific results related to MPTK are available in PDF format through the following page (https://gforge.inria.fr/docman/?group\_id=36)

\vspace{0.2 cm}

\noindent \underline{\emph{If you want specific information :}}

\vspace{0.2 cm}

You can write to us to matching.pursuit@irisa.fr.


\begin{center}
	\Large{\textbf{Request for help sent to this address won't be answered. Please use the Help forum instead.}}
\end{center}
\noindent \textbf{Thank you for your interest in The Matching Pursuit ToolKit ! }

