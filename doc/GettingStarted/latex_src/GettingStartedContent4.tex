%--------------------------------------------------------------
%--------------------------------------------------------------
% Section  5 : MPTK from within Matlab
%--------------------------------------------------------------
%--------------------------------------------------------------
\chapter{MPTK from within Matlab \label{MptkWithMatlab}}

In the following, we assume that MPTK has been installed correctly, that the system path configuration has been set and that Matlab is installed.

%--------------------------------------------------------------
% Section  5.1 : Getting Started
%--------------------------------------------------------------
\section{Getting Started}

\textcolor[rgb]{0.4,0.4,0.4}{\emph{\textbf{GettingStarted} command is used for a better understanding of MPTK functionalities. This script is divided in two parts. 
The first part consists in retrieving the environment informations and the available plugins. The second part consists in describing several tutorials about MPTK utilities.}}

\vspace{0.3 cm}

\noindent \underline{\emph{1st part description : Example of available plugins and dictionaries}}

\noindent Here is the list of types of atoms available in MPTK plugins:
\begin{center}
	\begin{tabular}{ll}
			- anywave	 & - anywavehilbert \\
			- constant	 & - dirac \\
			- gabor	 & - harmonic \\
			- mclt	 & - mdct \\
			- mdst	 & - nyquist \\
	\end{tabular}
\end{center}

\noindent As well as information on the path where reference files can be found:
\begin{center}
	- \emph{path\_to\_MPTK/mptk/reference}
\end{center}

which can be used to find examples of dictionaries:
\begin{center}
	\begin{tabular}{lll}
			dic\_anywave.xml & dic\_constant.xml & dic\_harmonic.xml \\
			dic\_mdst.xml & dic\_anywave\_modifie.xml & dic\_dirac.xml \\
			dic\_mclt.xml & dic\_nyquist.xml & dic\_chirp.xml \\
			dic\_gabor\_two\_scales.xml & dic\_mdct\_two\_scales.xml & dic\_test.xml \\
	\end{tabular}
\end{center}

\vspace{0.3 cm}

\noindent \underline{\emph{2nd part description : Description of the tutorials}}

\noindent There are several tutorials on using MPTK4Matlab:
\begin{my_enumerate}
	\item Dictionaries
	\item Books
	\item Running Matching Pursuit Toolkit
	\item Multichannel decompositions (in preparation) 
	\item Anywave atoms (in preparation)
	\item Demixing pursuit (in preparation)
\end{my_enumerate}

\noindent \underline{\emph{Dictionaries :}} How to read (\emph{dictread}), create (\emph{dictwrite}) a dictionary description \newline
\noindent \underline{\emph{Books :}} What is a book (storage format for sparse signal representations) and how to read 
(\emph{bookread}), save (\emph{bookwrite}) or plot any book (\emph{bookplot}, \emph{bookover}) \newline
\noindent \underline{\emph{Running MPTK :}} Procedure to follow if you want to decompose a signal :
\begin{my_itemize}
	\item Read a signal (\emph{sigread})
	\item Read a dictionary (\emph{dictread}) 
	\item Decompose the signal (\emph{mpdecomp})
\end{my_itemize}

%--------------------------------------------------------------
% Section  5.2 : Getting the environment information
%--------------------------------------------------------------
\section{Getting the environment information}

\textcolor[rgb]{0.4,0.4,0.4}{\emph{\textbf{getmptkinfo} command is launched under the GettingStarted.m script, and is used to retrieve 
the environment datas, such as : }}

\begin{my_itemize}
	\item \textcolor[rgb]{0.4,0.4,0.4}{The plugins atoms names available}
	\item \textcolor[rgb]{0.4,0.4,0.4}{The block parameters needed to correctly define each atom}
	\item \textcolor[rgb]{0.4,0.4,0.4}{The windows names that can be used to through the signal waveform}
	\item \textcolor[rgb]{0.4,0.4,0.4}{The example or default environment paths}
\end{my_itemize}

\noindent $\mapsto$ mptkInfo = getmptkinfo();

%--------------------------------------------------------------
% Section  5.3 : Reading a signal
%--------------------------------------------------------------
\section{Reading a signal}

\textcolor[rgb]{0.4,0.4,0.4}{\emph{\textbf{sigread} command reads an imports signal ``exampleSignal'' of any format supported by  libsndfile library 
to Matlab and gives a matrix ``signal'' (numSamples x numChans) and the sampling frequency of the read signal ``sampleRate''.}}

\vspace{0.2 cm}

\noindent $\mapsto$ [signal sampleRate] = sigread(mptkInfo.path.exampleSignal);

%--------------------------------------------------------------
% Section  5.4 : Reading a dictionary
%--------------------------------------------------------------
\section{Reading a dictionary}

\textcolor[rgb]{0.4,0.4,0.4}{\emph{\textbf{dictread} command imports a dictionary description ``defaultDict'' to Matlab and gives a dictionary 
description with the following structure :
		dict.block\{i\} = block       where, for example
			block.type = `dirac'}}

\vspace{0.2 cm}

\noindent $\mapsto$ dict = dictread(mptkInfo.path.defaultDict);

%--------------------------------------------------------------
% Section  5.5 : Decomposing a signal
%--------------------------------------------------------------
\section{Decomposing a signal}

\textcolor[rgb]{0.4,0.4,0.4}{\emph{\textbf{mpdecomp} command decompose a signal ``Signal'' using its sampling frequency ``sampleRate'', 
a dictionary structure ``dict'' performing ``numIter'' iterations and  gives the resulting decomposition ``book'', the ``residual'' obtained after 
the iterations and ``decay'', a vector with the energy of the residual after each iteration.}}

\vspace{0.2 cm}

\noindent $\mapsto$ [book residual decay] = mpdecomp(signal,sampleRate,dict,numIter);

%--------------------------------------------------------------
% Section  5.6 : Plotting a book
%--------------------------------------------------------------
\section{Plotting a book}

\textcolor[rgb]{0.4,0.4,0.4}{\emph{\textbf{bookover} plots the given ``book'' over a STFT spectrogram of the given ``Signal'' for  channel ``numChan'' 
(or 1 for default). The book and/or the signal can be given as  filenames (WAV format for the signal).}}

\vspace{0.2 cm}

\noindent $\mapsto$ figure(5);bookover(book,signal);

%--------------------------------------------------------------
% Section  5.7 : Reconstructing a signal
%--------------------------------------------------------------
\section{Reconstructing a signal}

\textcolor[rgb]{0.4,0.4,0.4}{\emph{\textbf{mprecons} reconstructs the signal from the given ``book''.}}

\vspace{0.2 cm}

\noindent $\mapsto$ sigrec = mprecons(book);

%--------------------------------------------------------------
% Section  5.8 : Finding further informations on MPTK for Matlab
%--------------------------------------------------------------
\section{Finding further informations on MPTK for Matlab}

\textbf{For further informations} on how to use MPTK Matlab functionalities please refer to the Chapter \ref{UseMatlabFunc} of the User Manual.