\chapter{pyMPTK: using MPTK with Python \label{GettingStartedPython}}

\begin{my_itemize}
\item decompose signals using MP or CMP, with a variety of atom types (gabor, chirp, anywave, ...)
\item reconstruct signals from books, including ones manipulated/generated from within python
\item plotting: a convenient method to plot gabor or chirped atoms
\end{my_itemize}

Some things not currently present in pyMPTK:
\begin{my_itemize}
\item Creating dictionary objects/files directly
\item Some decomposition methods: GMP, LoCOMP
\end{my_itemize}

System requirements:
\begin{my_itemize}
	\item Python (we tested with 2.7) with the following additional modules:
\begin{my_itemize}
\item NumPy
\item Matplotlib (if you want to plot e.g. using "mptkplot")
\end{my_itemize}
\end{my_itemize}

\section{Installation}

When you install MPTK, the CMake option BUILD\_PYTHON\_FILES is ON by default.
So, if you have the above system requirements, you should be ready to go.

TODO

\section{Quick guide}
Quick summary of the API:

\begin{verbatim}
	import mptk
	mptk.loadconfig(configpath)
	(book, residual) = mptk.decompose(signal, dictpath, samplerate)
	signal_remade    = mptk.reconstruct(book, dictpath)

	import mptkplot
	mptkplot.plot_chirped(book, samplerate)
\end{verbatim}

Remember the ``loadconfig'' command mentioned above -- it is needed for MPTK
to know where its plugins (etc) are located.

For fuller examples, please look in the "python" folder in your MPTK install path, for some example scripts.

Or after running ``import mptk'', use the command ``help(mptk)''
(the contents of that help are included later in this manual, Section \ref{UsePythonFunc}).

